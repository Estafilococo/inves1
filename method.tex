% !TeX root = main.tex

%%%
 % File: /latex/big-cocluster-paper/method.tex
 % Created Date: Monday, December 18th 2023
 % Author: Zihan
 % -----
 % Last Modified: Wednesday, 27th December 2023 12:01:16 pm
 % Modified By: the developer formerly known as Zihan at <wzh4464@gmail.com>
 % -----
 % HISTORY:
 % Date      		By   	Comments
 % ----------		------	---------------------------------------------------------
%%%

\section{Method}

In this section, we first introduce the problem of coclustering on large-scale data. Then we propose our method to solve this problem.

\subsection{Method Overview}

Our approach introduces a novel methodology for co-clustering that is primarily focused on enhancing scalability, adaptability and comprehensiveness in identifying co-clusters within bi-dimensional data. This is achieved through a three-step process: randomly partitioning the data matrix, co-clustering the resulting submatrices, and aggregating the co-clusters. Additionally, our method incorporates a probabilistic framework to determine the optimal number of matrix partitioning iterations, aiming to strike a balance between computational efficiency and thoroughness in pattern discovery. 

The process begins with the strategic partitioning of the data matrix. This initial partitioning is designed to break down the larger, more complex problem into smaller, more manageable sub-problems. This not only facilitates easier computational handling but also increases the potential for uncovering all possible co-clusters within the data.

Following the partitioning, we employ an ensemble method for co-clustering. In this stage, all co-clusters found within the individual submatrices are aggregated. This ensemble approach ensures that the overall clustering results are comprehensive and account for potential co-clusters that might be overlooked if the data were not partitioned.

A unique feature of our methodology is the incorporation of a probabilistic model to optimize the frequency of matrix partitioning. This model is predicated on achieving a predetermined probability of detecting all co-clusters within the data. Doing so allows for a dynamic adjustment in the number of partitions based on the specific characteristics and demands of the dataset, ensuring a balance between computational resources and the completeness of the co-clustering results.

In the subsequent sections, we will delve deeper into the specifics of the matrix partitioning technique, the ensemble co-clustering method, and the probabilistic framework. This will include a discussion of the algorithms employed, the rationale behind the probabilistic model, and the optimization strategies to enhance the scalability and efficacy of our approach.

\subsection{Notation and Definitions}
% Definitions of symbols, notations, and terminologies used in the method.
In this section, we define the notation and mathematical formulations used in our method. Understanding these definitions is crucial for comprehending the subsequent descriptions of the algorithm and its implementation.

\begin{itemize}
    \item $A \in \mathbb{R}^{M \times N}$ is a matrix with $K$ co-clusters (co-cluster set $C = \{C_k\}_{k=1}^K$);
    \item $A$ is partitioned into $m \times n$ blocks, each block having a size of $\phi_i \times \psi_j$, implying that $M=\sum_{i=1}^m \phi_i$ and $N=\sum_{j=1}^n \psi_j$;
    \item The block set is denoted as $B = \{B_{(i,j)}\}_{i=1}^m,_{j=1}^n$;
    \item The size of co-cluster $C_k \in \mathbb{R}^{M^{(k)} \times N^{(k)}}$ within block $B_{(i,j)}$ is represented as $M_{(i,j)}^{(k)} \times N_{(i,j)}^{(k)}$;
    \item $T_m$ represents the minimum number of rows, while $T_n$ denotes the minimum number of columns in a co-cluster.
\end{itemize}

With these notations established, we can now proceed to detail the algorithm and its components in the subsequent sections.

\subsection{Problem Formulation}
The task at hand involves the analysis of a data matrix, denoted as $A \in \mathbb{R}^{M \times N}$, comprising $M$ rows (representing features or variables) and $N$ columns (representing objects or samples). The entry at the $(i, j)$-th position of $A$ is given by $a_{ij}$, where $a_{i,\cdot}$ and $a_{\cdot,j}$ represent the $i$-th row and $j$-th column of $A$, respectively. 

A co-cluster within this context is identified as a subset of rows and/or columns that exhibit similar characteristics across a subset of columns and/or rows. Let $I$ be a set of row indices forming a row cluster, and $J$ be a set of column indices forming a column cluster; then the sub-matrix $A_{I,J}$ corresponds to a co-cluster in $A$. The objective of co-clustering is to partition the $M$ rows into $k$ row clusters and the $N$ columns into $d$ column clusters, resulting in a total of $k \times d$ co-clusters.

These partitions can be represented by a row label vector $u \in \mathbb{N}^M$ (where $u_i \in \{1, \ldots, k\}, \forall i \in \{1, \ldots, M\}$) and a column label vector $v \in \mathbb{N}^N$ (where $v_j \in \{1, \ldots, d\}, \forall j \in \{1, \ldots, N\}$). Indicator matrices $R \in \mathbb{R}^{M \times k}$ (where $\sum_{k'=1}^{k} R_{i,k'} = 1$) and $C \in \mathbb{R}^{N \times d}$ (where $\sum_{d'=1}^{d} C_{j,d'} = 1$) denote the row and column clusters. 

We aim to identify co-clusters $A_{I,J}$ in $A$ that are correlated according to specific criteria. These criteria include co-clusters with constant values, constant values along rows or columns, and co-clusters with additive or multiplicative coherent values. Recognizing and categorizing these patterns in $A$ forms the basis of our co-clustering approach, which is designed to effectively and efficiently identify such structures within the data.

\subsection{Matrix Partitioning}
% Description of the matrix partitioning process and criteria for partitioning.

\subsection{Co-Clustering within Submatrices}
% Detailed method for co-clustering within each of the submatrices.

\subsection{Probabilistic Framework}
% Presentation of the probabilistic model for determining the number of matrix partitioning iterations.

\subsection{Ensemble Method}
% Explanation of how co-clustered results from submatrices are aggregated.

\subsection{Algorithmic Description}
% Step-by-step algorithmic description of the process in pseudocode.

\subsection{Parameter Selection and Optimization}
% Discussion on the selection and optimization of parameters.

\subsection{Computational Complexity Analysis}
% (Optional) Analysis of the computational complexity of the method.

\subsection{Theoretical Justification or Rationale}
% Theoretical support or rationale for the effectiveness of the method.

\subsection{Potential Limitations and Assumptions}
% Acknowledgment of any limitations or assumptions in the method.
\subsection{Co-cluster}
Clustering is very popular and useful in multiple areas. However, co-clustering is more and more important.
Given two sets of objects $\mathcal{X} = \{x_1, x_2, \dots, x_n\}$ and $\mathcal{Y} = \{y_1, y_2, \dots, y_m\}$, a co-clustering algorithm simultaneously partitions $\mathcal{X}$ and $\mathcal{Y}$ into $k$ and $l$ clusters, respectively. The result is a $k \times l$ matrix $Z$ where $Z_{ij}$ is the cluster assignment of $x_i$ and $y_j$.
This is equivalent to finding a block structure of the matrix $Z$, i.e., finding $k \times l$ submatrices of $Z$ such that the elements within each submatrix are similar to each other and elements from different submatrices are dissimilar to each other.

\subsection{Analysis}

\subsubsection{Notation}


\subsubsection{Probability}
% P(Mi<Tm, Ni<Tn for all i in partition j | Qm, Qn, Ms, Ns)
Consider co-cluster $C_k$,
% P(M_{(i,j)}^{(k)} = i)
\begin{align*}
    P(M_{(i,j)}^{(k)} = \alpha) & = \frac{\binom{M^{(k)}}{\alpha} \binom{M-M^{(k)}}{\phi_i-\alpha}}{\binom{M}{\phi_i}} \\
    % p[x_] := Binomial[Mk, a] * Binomial[M-Mk, P-a] / Binomial[M, P]
    P(N_{(i,j)}^{(k)} = \beta)  & = \frac{\binom{N^{(k)}}{\beta} \binom{N-N^{(k)}}{\psi_j-\beta}}{\binom{N}{\psi_j}}
    % q[y_] := Binomial[Nk, b] * Binomial[N-Nk, Q-b] / Binomial[N, Q]
\end{align*}
The tail probability of $M_{(i,j)}^{(k)}$ and $N_{(i,j)}^{(k)}$ are
\begin{align*}
    P(M_{(i,j)}^{(k)} < T_m) & = \sum_{\alpha=1}^{T_m-1} P(M_{(i,j)}^{(k)} = \alpha) \\
                             & \le \exp(-2 (s_i^{(k)})^2 \phi_i)
\end{align*}
where $s_i^{(k)} = \cfrac{M^{(k)}}{M}-\cfrac{T_m-1}{\phi_i}$, and
\begin{align*}
    % &\le \sum_{\alpha=1}^{T_m-1} \frac{\binom{M^{(k)}}{\alpha} \binom{M-M^{(k)}}{\phi_i-\alpha}}{\binom{M}{\phi_i}} \\
    P(N_{(i,j)}^{(k)} < T_n) & = \sum_{\beta=1}^{T_n-1} P(N_{(i,j)}^{(k)} = \beta) \\
                             & \le \exp (-2 (t_j^{(k)})^2 \psi_j)
\end{align*}
where $t_j^{(k)} = \cfrac{N^{(k)}}{N}-\cfrac{T_n-1}{\psi_j}$.

The joint probability of $M_{(i,j)}^{(k)}$ and $N_{(i,j)}^{(k)}$ are
\begin{align*}
    P(M_{(i,j)}^{(k)} < T_m, N_{(i,j)}^{(k)} < T_n) & = \sum_{\alpha=1}^{T_m-1} \sum_{\beta=1}^{T_n-1} P(M_{(i,j)}^{(k)} = \alpha) P(N_{(i,j)}^{(k)} = \beta) \\
    % pq[x_, y_] := Sum[p[a] * q[b], {a, 1, x-1}, {b, 1, y-1}]
                                                    & \le \exp[-2 (s_i^{(k)})^2 \phi_i + -2 (t_j^{(k)})^2 \psi_j]
\end{align*}
If $\phi_i = p$ and $\psi_j = q$ for all $i$ and $j$, then

Suppose event $\omega_k$ is that co-cluster $C_k$ can't be find in any block $B_{(i,j)}$, then
\begin{align*}
    P(\omega_k) & = \prod_{i=1}^m \prod_{j=1}^n P(M_{(i,j)}^{(k)} < T_m, N_{(i,j)}^{(k)} < T_n)                          \\
                & \le \prod_{i=1}^m \prod_{j=1}^n \exp\{-2 \left[ (s_i^{(k)})^2 \phi_i + (t_j^{(k)})^2 \psi_j \right] \} \\
                & = \exp\{-2 \sum_{i=1}^m \sum_{j=1}^n \left[ (s_i^{(k)})^2 \phi_i + (t_j^{(k)})^2 \psi_j \right] \}     \\
\end{align*}

If $\phi_i = \phi$ and $\psi_j = \psi$ for all $i$ and $j$, then
\begin{align*}
    s_i^{(k)} & = s^{(k)} = \frac{M^{(k)}}{M}-\frac{T_m-1}{\phi} \\
    t_j^{(k)} & = t^{(k)} = \frac{N^{(k)}}{N}-\frac{T_n-1}{\psi}
\end{align*}

\begin{align*}
    P(\omega_k) & \le \exp \left\{ -2 [\phi m (s^{(k)})^2 + \psi n (t^{(k)})^2] \right\} \\
\end{align*}


And if we do $T_p$ times of random sampling, the Probability of detecting the co-cluster is
\begin{align*}
    P & = 1 - P(\omega_k)^{T_p}                                                        \\
      & \ge 1 - \exp \left\{ -2 T_p [\phi m (s^{(k)})^2 + \psi n (t^{(k)})^2] \right\} \\
\end{align*}
according to which, we can set $m, n, \phi, \psi, T_m, T_n$ and $T_p$ to ensure the probability of detecting the co-cluster is larger than a given threshold.

\subsubsection{Nosiy case}
\subsubsubsection{Assumption}
Assume each noise $n_{ij}$ complies with a normal distribution $N(0, \sigma^2)$, i.i.d. for all $i$ and $j$. Suppose $\exists \lambda > 0$, such that
$$\lambda \le \max(||B||_1, ||B^\top||_1)/\sigma^2.$$

\subsubsubsection{Score}
The score of a submatrix $A_{I,J}$ is defined as
\begin{align}
    S(I,J)       & = \min(S_{row}(I,J), S_{col}(I,J))                                                                                          \\
    S_{row}(I,J) & = \min_{i_1, i_2 \in I} \left(1- \frac{1}{|I|-1} \sum_{i_2 \in I, i_2 \neq i_1} \langle x_{i_1,J}, x_{i_2,J}\rangle \right) \\
    S_{col}(I,J) & = \min_{j_1, j_2 \in J} \left(1- \frac{1}{|J|-1} \sum_{j_2 \in J, j_2 \neq j_1} \langle x_{I,j_1}, x_{I,j_2}\rangle \right)
\end{align}
where $x_{i,J}$ is the $i$-th row of $A_{I,J}$. Here we define the inner product of two vectors $x$ and $y$ as
$$\langle x, y \rangle = \exp(-\frac{||x - y||_1^2}{2\alpha||x||_1||y||_1})$$

Suppose $A$ is a hidden co-cluster matrix, and $E$ is the noise matrix. Then the observed matrix is $B = A + E$. Consider co-cluster $B_{I,J}$, denote $1 - \frac{1}{|I|-1} \sum_{i_2 \in I, i_2 \neq i_1} \langle x_{i_1,J}, x_{i_2,J}\rangle$ as $s_{row}(i_1, i_2, J)$, then
\begin{align*}
    \mathbb{E}(s_{row}(i_1, i_2, J)) & = 1 - \frac{1}{|I|-1} \sum_{i_2 \in I, i_2 \neq i_1} \mathbb{E}(\langle x_{i_1,J}, x_{i_2,J}\rangle)                                    \\
                                     & = 1 - \frac{1}{|I|-1} \sum_{i_2 \in I, i_2 \neq i_1} \exp(-\frac{||x_{i_1,J} - x_{i_2,J}||_1^2}{2\alpha||x_{i_1,J}||_1||x_{i_2,J}||_1}) \\
                                     & \ge 1 - \exp(-\frac{2}{\alpha \min(||x_{i_1,J}||_1, ||x_{i_2,J}||_1)})                                                                  \\
                                     & \ge 1 - \exp(-\frac{2}{\alpha \max(||B||_1, ||B^\top||_1)})                                                                             \\
    \sigma^2(s_{row}(i_1, i_2, J))   & = \frac{1}{|I|-1} \sum_{i_2 \in I, i_2 \neq i_1} \sigma^2(\langle x_{i_1,J}, x_{i_2,J}\rangle)                                          \\
                                     & = \frac{1}{|I|-1} \sum_{i_2 \in I, i_2 \neq i_1} \exp(-\frac{||x_{i_1,J} - x_{i_2,J}||_1^2}{2\alpha||x_{i_1,J}||_1||x_{i_2,J}||_1})     \\
                                     & \le \sigma^2 \exp(-\frac{2}{\alpha \min(||x_{i_1,J}||_1, ||x_{i_2,J}||_1)})                                                             \\
                                     & \le \sigma^2 \exp(-\frac{2}{\alpha \max(||B||_1, ||B^\top||_1)})                                                                        \\
\end{align*}

Thus the expected value of $S_{row}(I,J)$ satisfies
\begin{align*}
    \mathbb{E}(S_{row}(I,J)) & \ge |J||I| \left(1 - \exp(-\frac{2}{\alpha \max(||B||_1, ||B^\top||_1)}) \right) \\
    \sigma^2(S_{row}(I,J))   & \le |I||J| \sigma^2 \exp(-\frac{2}{\alpha \max(||B||_1, ||B^\top||_1)})          \\
\end{align*}

Similarly, we can get
\begin{align*}
    \mathbb{E}(S_{col}(I,J)) & \ge |J||I| \left(1 - \exp(-\frac{2}{\alpha \max(||B||_1, ||B^\top||_1)}) \right) \\
    \sigma^2(S_{col}(I,J))   & \le |I||J| \sigma^2 \exp(-\frac{2}{\alpha \max(||B||_1, ||B^\top||_1)})          \\
\end{align*}

Then since $x^2 \le x$ for $x \in (0,1)$, we have
\begin{align*}
    \mathbb{E}(S(I,J)) & \ge 1 - \exp(-\frac{2}{\alpha \max(||B||_1, ||B^\top||_1)})             \\
    \sigma^2(S(I,J))   & \le |I||J| \sigma^2 \exp(-\frac{2}{\alpha \max(||B||_1, ||B^\top||_1)}) \\
\end{align*}

According to the Chernoff bound, we have
\begin{align*}
    P(S(I,J) \le \mathbb{E}(S(I,J)) - \epsilon)
     & \ge \exp(-\frac{\epsilon^2}{2|I||J| \sigma^2 \exp(-\frac{2}{\alpha \lambda})}) \\
\end{align*}

Thus if we set $\epsilon = \sqrt{2|I||J| \sigma^2 \exp(-\frac{2}{\alpha \lambda}) \log(1/\delta)}$, then

\begin{align*}
    P(S(I,J) \ge \mathbb{E}(S(I,J)) - \epsilon) & \ge \delta \\
\end{align*}

Combined with the probability control of $T_p$, we can select parameters to ensure the probability of detecting the co-cluster is larger than a given threshold.

\begin{algorithm}[!t]
    \caption{Probabilistic Ensemble Co-Clustering Method}\label{alg:method}
    \begin{algorithmic}[1]
        \REQUIRE{Data matrix $A \in \mathbb{R}^{M \times N}$, Co-cluster set $C = \{C_k\}_{k=1}^K$, Block sizes $\{\phi_i\}_{i=1}^m$, $\{\psi_j\}_{j=1}^n$, Thresholds $T_m$, $T_n$, Sampling times $T_p$, Probability threshold $P_{thresh}$;}
        \ENSURE{Co-clustered result $\mathcal{C}$;}
        \STATE Initialize block set $B = \{B_{(i,j)}\}_{i=1}^m,_{j=1}^n$ based on $\phi_i$ and $\psi_j$
        \STATE Calculate $s^{(k)}$ and $t^{(k)}$ for each co-cluster $C_k$
        \FOR{$k=1$ to $K$}
            \STATE Calculate $P(\omega_k)$ for co-cluster $C_k$
            \IF{$P(\omega_k) < P_{thresh}$}
                \STATE Partition matrix $A$ into blocks $B$ and perform co-clustering
                \STATE Aggregate co-clustered results from each block
            \ENDIF
        \ENDFOR
        \RETURN Aggregated co-clustered result $\mathcal{C}$
    \end{algorithmic}
\end{algorithm}