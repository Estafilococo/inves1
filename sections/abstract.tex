%%%
 % File: /latex/big-cocluster-paper/abstract.tex
 % Created Date: Tuesday, January 23rd 2024
 % Author: Zihan
 % -----
 % Last Modified: Tuesday, 23rd January 2024 11:19:03 am
 % Modified By: the developer formerly known as Zihan at <wzh4464@gmail.com>
 % -----
 % HISTORY:
 % Date      		By   	Comments
 % ----------		------	---------------------------------------------------------
%%%

\begin{abstract}
	This paper presents an innovative approach to co-clustering, a crucial technique in data analysis for revealing intricate patterns within high-dimensional and sparse data structures. Recognizing the limitations of traditional co-clustering methods, including time-intensive processing, high memory usage, and uncertainty in identifying all relevant co-clusters, we introduce a novel methodology that integrates matrix partitioning with an ensemble method. This approach significantly reduces computational complexity and enhances the detection of nuanced patterns by partitioning data matrices into smaller submatrices before applying the ensemble method. A key innovation is the introduction of a probabilistic framework to optimize matrix partitioning, striking a balance between computational efficiency and analytical thoroughness. Our extensive evaluation demonstrates the method's effectiveness and adaptability across various domains, including text analysis, biomedical data analysis, and financial pattern recognition. By integrating this approach with longitudinal biclustering models, we pave the way for faster identification of stable co-clusters, contributing to the advancement of efficient and scalable biclustering techniques in the rapidly evolving field of data analysis.
\end{abstract}