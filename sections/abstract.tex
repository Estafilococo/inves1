%%%
 % File: /latex/big-cocluster-paper/abstract.tex
 % Created Date: Tuesday, January 23rd 2024
 % Author: Zihan
 % -----
 % Last Modified: Wednesday, 20th March 2024 12:51:30 pm
 % Modified By: the developer formerly known as Zihan at <wzh4464@gmail.com>
 % -----
 % HISTORY:
 % Date      		By   	Comments
 % ----------		------	---------------------------------------------------------
%%%

\begin{abstract}
This paper presents a novel co-clustering method involving dynamic partition and hierarchical ensembling, which is a crucial technique in data analysis for uncovering complex patterns within high-dimensional and large-scale data structures. Due to the limitations of traditional co-clustering methods in big data, including time-intensive processing, high memory usage, and uncertainty in identifying all relevant co-clusters, we propose a new method combining matrix partitioning and ensemble. This approach significantly reduces computational complexity and enhances the detection of nuanced patterns by partitioning data matrices into smaller sub-matrices before applying the ensemble method. A key contribution is the introduction of a probabilistic framework to achieve dynamic the process of matrix partitioning, striking a balance between computational efficiency and analytical thoroughness. Our extensive evaluation demonstrates the method's effectiveness and adaptability across various domains, including text analysis, biomedical data analysis, and financial pattern recognition. By integrating this approach with longitudinal co-clustering models, we pave the way for faster identification of stable co-clusters, thereby contributing to the advancement of efficient and scalable co-clustering techniques in the rapidly evolving field of data analysis.
\end{abstract}

% keywords can be removed
% \keywords{First keyword \and Second keyword \and More}